\documentclass[ignorenonframetext,]{beamer}
\setbeamertemplate{caption}[numbered]
\setbeamertemplate{caption label separator}{:}
\setbeamercolor{caption name}{fg=normal text.fg}
\usepackage{amssymb,amsmath}
\usepackage{ifxetex,ifluatex}
\usepackage{fixltx2e} % provides \textsubscript
\usepackage{lmodern}
\ifxetex
  \usepackage{fontspec,xltxtra,xunicode}
  \defaultfontfeatures{Mapping=tex-text,Scale=MatchLowercase}
  \newcommand{\euro}{€}
\else
  \ifluatex
    \usepackage{fontspec}
    \defaultfontfeatures{Mapping=tex-text,Scale=MatchLowercase}
    \newcommand{\euro}{€}
  \else
    \usepackage[T1]{fontenc}
    \usepackage[utf8]{inputenc}
      \fi
\fi
% use upquote if available, for straight quotes in verbatim environments
\IfFileExists{upquote.sty}{\usepackage{upquote}}{}
% use microtype if available
\IfFileExists{microtype.sty}{\usepackage{microtype}}{}
\usepackage{longtable,booktabs}
\usepackage{caption}
% These lines are needed to make table captions work with longtable:
\makeatletter
\def\fnum@table{\tablename~\thetable}
\makeatother

% Comment these out if you don't want a slide with just the
% part/section/subsection/subsubsection title:
\AtBeginPart{
  \let\insertpartnumber\relax
  \let\partname\relax
  \frame{\partpage}
}
\AtBeginSection{
  \let\insertsectionnumber\relax
  \let\sectionname\relax
  \frame{\sectionpage}
}
\AtBeginSubsection{
  \let\insertsubsectionnumber\relax
  \let\subsectionname\relax
  \frame{\subsectionpage}
}

\setlength{\parindent}{0pt}
\setlength{\parskip}{6pt plus 2pt minus 1pt}
\setlength{\emergencystretch}{3em}  % prevent overfull lines

\date{}

\begin{document}

\begin{frame}\frametitle{}

\ctable[pos = H, center, botcap]{l}
{% notes
}
{% rows
\FL
title: A better webcrawler in Go
\\\noalign{\medskip}
subtitle: Multimedia Information Retrieval
\\\noalign{\medskip}
author: Neal van Veen
\\\noalign{\medskip}
header-includes:
\\\noalign{\medskip}
- \usepackage{cleveref}
\\\noalign{\medskip}
date: \today
\LL
}

\end{frame}

\begin{frame}\frametitle{Advantages in Go}

\begin{itemize}
\item
  Static compiled
\item
  Rapid deployment (think Docker)
\item
  Very simple and easy to learn
\item
  Concurrency primitives
\item
  Tooling
\end{itemize}
\end{frame}

\begin{frame}\frametitle{Disadvantages}

\begin{itemize}
\item
  Not very flexible in syntax
\item
  Sometimes a bit too simple
\item
  Not as safe as needed when dealing with concurrency
\item
  Not as fast as C/C++
\item
  Language is still very young, needs improvements
\end{itemize}
\end{frame}

\begin{frame}\frametitle{Challenges}

\begin{itemize}
\item
  Aforementioned safety is a problem
\item
  Hashmaps used are not thread-safe, deadlocking remains an issue
\item
  New way of thinking when programming for concurrency
\item
  No real idiomatic way of handling errors
\end{itemize}
\end{frame}

\begin{frame}\frametitle{Data structures}

\begin{itemize}
\item
  Implemented simple linked list, binary tree and fragment tree.
\item
  Fragment-tree does not conform to Container interface, so haven't used
  it yet
\item
  Using a two thousand element-big hashmap for keeping track of
  robots.txt
\item
  When it is full, it is cleared, so we can somewhat buffer the contents
  without storing everything
\item
  Developed a unit-testing framework for using a MongoDB backend
\item
  No testing done for actual communication with backend over a network
\item
  Also use a mock storage (a simple hashmap) for unittesting, that
  conforms to the same Storage interface.
\end{itemize}
\end{frame}

\begin{frame}\frametitle{Results}

\begin{itemize}
\item
  Developed a simple Go program that generates links on a local network
  with a 20ms delay, so we don't pester certain websites (the LIACS
  websites, for instance) with too much testing HTTP requests.
\item
  Sequential crawling of 5 minutes for 2000 requests.
\item
  Concurrent crawling with 10 goroutines of 40 seconds for 2000
  requests.
\item
  Concurrent crawling with 100 goroutines of 6 seconds for 2000
  requests.
\item
  More goroutines testing planned, running on better hardware.
\item
  Will be running a crawling session with a large (1000ish, hopefully)
  on the open net.
\item
  Storage requirements for datastructures haven't been analyzed yet.
\end{itemize}
\end{frame}

\end{document}
